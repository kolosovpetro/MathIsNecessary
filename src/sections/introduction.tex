Начнем с истоков.
Когда человек задает не совсем корректный вопрос вида ''Нужна ли математика'' -- важно понимать что, на самом деле,
с точки зрения вопрошающего, вопрос более чем корректный и причина ему банальный когнитивный диссонанс.
Механизм когнитивного диссонанса будет раскрыт чуточку позже.
Сейчас же, давайте рассмотрим мотивацию человека, ставящего вопрос о необходимости математики в жизни каждого.

Действительно, какая нахрен математика, когда у меня три недосмотренных сериала по нетфликсу, два открытых на амазон прайм
во вкладках хрома, а в холодильнике лежит недобитая двушка пива "Толстяк крепкое"\footnote{Извиняюсь за рекламу},
завтра мне вставать на завод, а моя жена ворчит что Васька купил Ленке клевую шубу за 5к дерева, че ты мне втираешь тут?

Совершенно справедливый вопрос, который вне всякого сомнения, возникнет в голове среднестатистического обывателя.
И такой обыватель, исходя из личного субъективного опыта, будет прав, подчеркиваю, будет прав.

Так что случилось-то?
Почему человек так радикально и решительно настроен против царицы всех наук, сведетельствующей о наличии истинной красоты
и грации в этом бренном мире?
Как так получилось-то?
Я более чем уверен что такой человек даже будет иметь какие-то свои понятия прекрасного, какого-то вкуса в культуре и искусстве,
но при этом десятой стороной обходящего одну из самых прекрасных наук.

Тут мы подходим к понятию когнитивного диссонанса в голове обывателя.
Вызван этот диссонанс, как ни странно, системой образования, которая волей не волей, подменила понятие ''математика'' понятием
''задачки для дебилов''.
Вся проблема заключается в том, что настоящая математика вообще незнакома многим людям.

\begin{displayquote}
    \textit{
        Математика -- это не задачки для дебилов в школе.
        Математика описывает закономерности и взаимоотношение между объектами.
        Математика, также, является языком описания реальности.
        Если угодно, через математику мы декомпилируем реальность.
    }
\end{displayquote}
Тобишь, на примере физики, мы видим как математическая модель отображает реальное поведение различных физических процессов.
Твое
\[F = ma\]
справедливо как для 5-килограммового пакета продуктов с пятерочки по акции, так и при летящем тебе в табло кулаке гопника,
отжимающего твой новенький айфон в кредит.
Теперь, я надеюсь, понятна суть того, чем является математика на самом деле.

\subsection{Принятие решений}\label{subsec:decision-making}
Все же, вернемся к более приземленным вещам, и покажем где и как именно математика играет свою роль в нашей с вами повседневной жизни.
Как ни странно, задача о распределении свободного времени и принятие решения ''чем же заняться'' -- это комбинаторная задача оптимизации,
о том как именно распределить свободное время с максимизацией желаемого результата,
будь то получение удовольствия, знаний и так далее.
Принятие решения ''чем же сейчас заняться'', добить баклашку толстяка или посмотреть нетфликс -- это \textbf{NP}-полная задача.
Забавно, да?
Вот, оказывается, наш подпивасник, даже не осознавая, решает \textbf{NP}-полную задачу каждый день,
когда принимает решение идти за добавкой в круглосуточный ларек.

\subsection{Алгоритмы выдачи соц сетей}\label{subsec:social-media}

Далее, рассмотрим наши с вами любимые социальные сети, поисковые системы, ютуб и тикток.
Все же любят ютуб и тикток?
Да, я знаю, ты тоже смотришь тиктоки.
Казалось бы, причем здесь математика?
И что же получается, оказывается, алгоритм выдачи социальных таких как ютуб или тик-ток основан на
цепях Маркова\footnote{Загугли}.
Цепи Маркова, в свою очередь, базируются на комбинации теории графов и теории вероятностей.
Так, согласно цепей Маркова, вероятность выдачи конкретной ноды графа определяется количеством входящих в эту
ноду ребер (ссылок).
Звучит немного сложно, поясню, чем больше ссылок на ноду, тем более вероятно эта нода попадет в предложку по следующему видео.
Ровно так же работает поисковая система гугл, чем больше ссылок на страницу -- тем более вероятно страница попадет в выдачу
по запросу.
Я говорю очень упрощенно, на самом деле алгоритмы выдачи гугла куда сложнее и захватывают
не только количество входящих ссылок на страницу, но также и телеметрию посетителей страницы,
время пребывания на странице и тд.
Но общую суть вы уловили.

Обыватель может возмутиться,

\begin{displayquote}
    \textit{
        И зачем мне понимать как работает гугл и ютуб, я смотрю себе свои тиктоки и видео и мне ок, а значит математика все еще не нужна
    }
\end{displayquote}

А затем, что если мы рассмотрим влияние алгоритмов выдачи контента, умных лент и прочего на человека, то внезапно (какая неожиданность),
мы заметим, что они (снова внезапно) всецело формируют мировозрение и картину мира человека.
Соц.\ сети также могут формировать подобие информационного пузыря, находясь в котором,
человек будет буквально вариться в однотипной информации и искренне верить что новости с ютуба
отображают действительное положение вещей в реальности.
И вот, получается, наполнение черепушки обывателя определяется дискретным алгоритмом выдачи, написанным командой разработчиков гугла?
Во дела!
И тебе все еще не нужна математика?
Не знаю будет ли для тебя секретом или откровением, но все ''твое'' мировоззрение определено алгоритмом на графах,
описанным задолго до твоего рождения.
Другими словами, ты в матрице, \texttt{\%username\%}.
А в матрице не будет лишним понимать законы и устройство этой самой матрицы.

В заключение, хотелось бы напомнить о позитивном влиянии математики на структурирование мышления человека.
Еще М.\ Ломоносов сказал

\begin{displayquote}
    \textit{
        Математику уже затем учить надо, что она ум в порядок приводит
    }
\end{displayquote}

Никто же не спрашивает о том, зачем нужна физкультура и занятие спортом, физическое здоровье же улучшает.
Так же и с математикой, физкультура и спорт для ума.
Кстати, Арнольд Шварцнеггер занимается математикой чтоб в старости не схлопотать болезнь альцгеймера.